\chapter{Conclusioni e sviluppi futuri}

Questo progetto ha dimostrato come integrare un chatbot all'interno di una web application già strutturata sia possibile anche se sono necessarie delle introduzioni di altre tecnologie non preesistenti. Nonostante le tecnologie usate non siano le più aggiornate, come ad esempio AngularJs \footnote{Che nel dicembre 2021 non verrà più mantenuto e ad oggi sono state create 11 versioni stabili dopo questa}, o le più facili da usare come il sistema di versionamento Subversion \footnote{Version Control system di tipo centralizzato, che ad oggi è stato sorpassato di gran lunga da Git che invece è di tipo distribuito}, è comunque possibile trovare degli workaround per poter introdurre nuove tecnologie all'interno di un software complesso come quello su cui ho dovuto lavorare.
L'utilizzo nella prima parte del progetto di Dialogflow è stato fondamentale per comprendere il tipo di software che era necessario creare, nonostante sia stato rimpiazzato poi da una tecnologia completamente differente e se vogliamo più complessa in termini di implementazione.
Il problema più grosso che ho riscontrato durante la creazione del prototipo è stato quello di non riuscire ad automatizzare il sistema di acquisizione dei token del cliente che però non sussiste nel caso in cui la chat sia già integrata, infatti in questo caso i due token possono essere inviati dal frontend non appena la chat viene aperta o non appena viene inviato il primo messaggio dal cliente.
Lo scopo finale del progetto è stato raggiunto portando una prima versione funzionante del chatbot richiesto. Gli sviluppi futuri di questa prima versione sono numerosi. Prima fra tutti è la necessità di migliorare la qualità delle stories che si trovano nel dataset di training dell'intelligenza in modo tale da rendere il modello più accurato e preciso. L'interfaccia della chat integrata al software è da rivedere aggiungendo migliorie grafiche studiate per rendere l'esperienza utente il più accogliente possibile. Infine il bot al momento ha solo le funzionalità base, ma è possibile implementare nuove funzionalità aggiornando il backend Rasa con tutte le operazioni che potrebbe fare un cliente della banca.
Il prototipo che ho creato è standalone e open source, l'intero codice infatti può essere visionato su Github \footnote{Servizio di hosting di progetti basato sulla tecnologia Git}, per questioni ovvie non si trova nessun riferimento diretto a codice e indirizzi del software proprietario aziendale.
