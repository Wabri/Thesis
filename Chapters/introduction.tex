\chapter{Introduzione}
Al giorno d'oggi l'utilizzo della voce per automatizzare i processi della vita quotidiana si sta sempre pi\'u espandendo, dall'assistente vocale nella propria casa a quello nel proprio cellulare. Sono sempre stato incuriosito da questo tipo di tecnologie e il tirocinio era il modo migliore per comprendere lo sviluppo di software di questo tipo. Il progetto mi \'e stato proposto dall'azienda fiorentina E.D.P. Service, societ\'a collegata al gruppo Best Vision Holding, il cui scopo era integrare una funzionalit\'a nuova nel sito del cliente BPS(suisse) che permettesse di eseguire pagamenti o trasferimenti di denaro usando linguaggio naturale comunicando direttamente con l'applicazione.
I pagamenti fatti tramite la piattaforma web utilizzano la classica procedura che di compilamento manuale di alcuni campi di testo (brutta brutta questa frase, riguarda), come ad esempio la quantit\'a di denaro da trasferire e l'iban del conto del destinatario.
Software di questo tipo si suddividono in 2 parti: una parte visibile all'utente con cui pu\'o interagire rappresentata dal frontend (chiamato anche interfaccia grafica) generalmente \'e l'interfaccia web, e la parte che gestisce le informazioni rappresentata dal backend (chiamata anche lato server).
La parte del frontend gestisce principalmente le modalit\'a di accesso dell'utente consentendogli di interagire con l'applicativo eseguendo le azioni messe a disposizione, come ad esempio: visualizzare il numero di transazioni eseguite, creare un pagamento, controllare l'andamento del mercato in base a una determinata valuta. 
Per riuscire a sviluppare un'integrazione ho studiato i linguaggi usati dalla piattaforma, principalmente formata da codice HTML, CSS e AngularJS, web-framework di casa Google scritto in JavaScript la cui prima versione fu rilasciata nel 2010. Questa libreria permette di dinamicizzare una pagina web utilizzando dei comandi a livello di codice HTML, chiamati direttive, che vengono poi eseguiti da AngularJS una volta che la pagina web viene caricata dal browser.
Tutto ci\'o che riguarda il lato server: gestione, elaborazione e immagazzinamento dei dati forniti dal frontend viene eseguita dal backend, parte fondamentale di un servizio web bancario che deve garantire sicurezza e reperibilit\'a dei dati. Il backend sviluppato da Best Vision \'e diviso in un servizio di database e uno di gestione del database scritto per intero in linguaggio Java usando il framework Spring. 
Inserire una funzionalit\'a che permettesse di eseguire queste azioni usando la sola voce avrebbe diminuito il numero di azioni che l'utente avrebbe dovuto eseguire per ottenere ci\'o che cercava, dovevo quindi imitare il comportamento umano di un funzionario (operatore,cassiere? non so quale suona meglio) di banca. Per fare questo ho pensato di creare un servizio di chat bot che riuscisse a comprendere il linguaggio naturale umano e producesse una risposta in completa autonomia eseguendo le azioni richieste dall'utente necessarie.
Il pagamento tramite chat doveva quindi eseguire tutte le richieste al backend nel giusto ordine prima di completare la transazione voluta: autorizzazione dell'utente alla gestione del conto, richiesta di trasferimento di capitale verso un altro utente e delega della conferma del pagamento.
Nel mio caso il frontend \'e rappresentato dalla chat in cui l'utente pu\'o decidere se scrivere un testo con linguaggio naturale oppure utilizzare il servizio vocale, il backend invece \'e composto da un'intelligenza artificiale che genera una risposta e agisce in base alle richieste.
Quando un utente invia un messaggio si aspetta che venga generata una risposta consona, cio\'e venga eseguita una analisi del testo e formulata una frase di senso compiuto che riesca a rispondere in maniera corretta al messaggio inviato dall'utente. La analisi del testo viene definita comprensione del linguaggio naturale, in inglese Natural Language Understanding (d'ora in poi lo indicher\'o con NLU), cio\'e attribuire alla frase analizzata un intento e estrarre da essa oggetti che possono essere utili per la generazione di una risposta. L'effettiva generazione della risposta per\'o non \'e un compito che viene esercitato dal NLU, ma da un tool esterno che in base all'analisi fornita riesca a determinare la miglior risposta possibile.
Il chatbot voluto \'e quindi formato dalla chat e 2 tipi di intelligenza artificiale: una che esegue il NLU e un'altra che genera la risposta.
Per la creazione del bot ho usato una libreria open source sviluppata in Python chiamata Rasa, che mette a disposizione tutti gli strumenti che servono per poter sviluppare intelligenze artificiali di questo tipo. I tools che ho usato sono: rasa\_nlu e rasa\_core, il primo per eseguire il compito di NLU e l'altro per generare una risposta e eseguire le azioni necessarie.
L'esecuzione di questi 2 strumenti si basa su dei modelli generati a partire da dei dataset creati precedentemente che devono contenere dei veri e propri esempi di analisi di possibili frasi che l'utente potr\'a inserire, un numero molto alto di esempi fornir\'a una precisione maggiore da parte dell'intelligenza.
Il dataset che ho usato non \'e particolarmente grande dato che riuscivo comunque a presentare un prototipo di bot ragionevolmente preciso anche con un dataset di dimensioni accettabili, infatti la popolazione del primo prototipo non superava i 60 esempi.
Completato lo sviluppo delle 2 intelligenze artificiali queste sono state inserite (non mi piace questo verbo, trovare un sinonimo adeguato) in un server dove potessero rimanere in ascolto di tutti i messaggi da parte degli utilizzatori della chat.
L'ultimo passo \'e stato quello di integrare il prototipo della chat nel sito web di BPS(suisse) attuando alcune modifiche per quanto riguarda gli indirizzi relativi alla trasmissione dei messaggi. 
(inserisci immagine qui: esempio rasainout).
Questo ha completato il lavoro di tirocinio consentendo all'azienda di inserire nella fase di produzione e miglioramento l'integrazione da me sviluppata.


<frase iniziale> 
<perchè ho scelto questa tesi>
<perchè edp ha scelto questo percorso>
<che progetto è?>
<spiegazione generica degli elementi fondamentali del progetto>
<definizione di un percorso iniziale>
<spiegazione dell'architettura del software di BPS(swisse)>
<studio della libreria principale usata dal frontend del software>
<studio degli elementi principali per la creazione del frontend ----> chat di testo>
<creazione della chat e test del prototipo usando sia testo che voce>
<studio di una possibile implementazione del backend di nlu>
<perchè è stato scelto di non usare dialogflow>
<passaggio a RASA>
<sviluppo del primo modello usato per il NLU>
<creazione del server rasa>
<refactor chat>
<perfezionamento del modello di NLU>
<integrazione>
<problemi affrontati durante l'integrazione>
<conclusione introduzione>
