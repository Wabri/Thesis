\chapter{Introduzione}
Al giorno d'oggi l'utilizzo della voce per automatizzare i processi della vita quotidiana si sta sempre pi\'u espandendo, dall'assistente vocale nella propria casa a quello nel proprio cellulare. Sono sempre stato incuriosito da questo tipo di tecnologie e il tirocinio era il modo migliore per comprendere lo sviluppo di software di questo tipo. Il progetto mi \'e stato proposto dall'azienda fiorentina E.D.P. Service, societ\'a collegata al gruppo Best Vision, il cui scopo era creare un prototipo di chat vocale per pagamenti bancari da integrare poi nel sito del cliente BPS(suisse).
L'azienda mi ha proposto un progetto per lo sviluppo di un software che permettesse di eseguire pagamenti vocali all'interno della piattaforma web della BPS(Suisse).
Software di questo tipo si suddividono in 2 parti: una parte visibile all'utente con cui pu\'o interagire rappresentata dal frontend (chiamato anche interfaccia grafica) e la parte che gestisce le informazioni rappresentata dal backend (chiamata anche lato server).
La parte del frontend gestisce principalmente le modalit\'a di accesso dell'utente consentendogli di interagire con l'applicativo per eseguire la azioni messe a disposizione, come ad esempio: visualizzare il numero di transazioni eseguite, creare un pagamento, controllare l'andamento del mercato in base a una determinata valuta. Inserire una chat che permettesse di eseguire queste azioni usando la sola voce avrebbe diminuito il numero di azioni che l'utente avrebbe dovuto eseguire per ottenere ci\'o che cercava. 
Per poterlo fare ho studiato i linguaggi usati dalla piattaforma che principalmente era formata da codice HTML, CSS e AngularJS, web-framework di casa Google scritto in JavaScript la cui prima versione fu rilasciata nel 2010. Questa libreria permette di dinamicizzare una pagina web utilizzando dei comandi, chiamati direttive, a livello di codice HTML che vengono poi eseguiti da AngularJS una volta che la pagina viene caricata.
Tutto ci\'o che riguarda il lato server: gestione, elaborazione e immagazzinamento dei dati forniti dal frontend, viene eseguita dal backend parte fondamentale di un servizio web bancario che deve garantire sicurezza e reperibilità dei dati. Il backend sviluppato da Best Vision \'e diviso in un servizio di database e uno di gestione del database scritto per intero in linguaggio Java usando il framework Spring. Il pagamento tramite chat doveva quindi eseguire tutte le richieste al backend nel giusto ordine prima di completare la transazione voluta: autorizzazione dell'utente alla gestione del conto, richiesta di trasferimento di capitale verso un altro utente e delega della conferma del pagamento.
Quando un utente invia un messaggio si aspetta che venga generata una risposta consona, cioè venga eseguita una analisi del testo e formulata una frase di senso compiuto che riesca a rispondere in maniera corretta al messaggio dell'utente. La comprensione del testo viene definita comprensione del linguaggio naturale, in inglese Natural Language Understanding (d'ora in poi lo indicherò con NLU), cio\'e attribuire alla frase analizzata un intento e estrarre da essa oggetti che possono essere utili per la generazione di una risposta.

Se il lavoro di comprensione e generazione della frase di risposta deve essere fatto da un computer questo non \'e semplice
Il problema principale della creazione di un bot \'e la creazione di un metodo di comprensione del testo che l'utente ha inviato



Per definire la fattibilit\'{a} di questo progetto ho studiato innanzi tutto l'architettura del software in uso con le annesse tecnologie principali: AngularJS e Spring.
Il passo successivo \'{e} stato quello di ricercare un metodo che permettesse di elaborare un testo preso in input e restituire in output una azione da compiere in base al contenuto dell'input. Le ricerche sono confluite in un particolare tipo di intelligenza artificiale chiamata Natural Language Understanding (che d'ora in poi indicher\'{o} con NLU), tradotto letteralmente comprensione del linguaggio naturale. 
Il NLU \`{e} definito come la comprensione della struttura e il significato da parte degli elaboratori del linguaggio umano (per esempio Inglese, Spagnolo, Giapponese) permettendo agli utenti di interagire con i computer usando frasi naturali.
Per la creazione effettiva del prototipo avevo quindi bisogno di un software che eseguisse il lavoro di NLU e una chat dove permettere uno scambio sia di tipo vocale che testuale con l'utente.
La prima versione dell'intelligenza artificiale \'e stata creata grazie allo strumento messo a disposizione da Google chiamato DialogFlow. Questo strumento mi ha consentito di creare un primo modello di NLU, comprendendo l'effettivo potenziale di un'intelligenza artificiale di questo tipo. 
Questo strumento per\'o non \'e stato scelto per lo sviluppo successivo per questioni di privacy, dato che i dati che sarebbero circolati nei server Google sarebbero stati dati di natura strettamente personale e monetaria.
Era quindi necessario un software da poter gestire e sviluppare internamente all'azienda, che fosse open source e in una versione stabile. Leggendo in vari forum e in vari articoli Rasa era quello che faceva al caso nostro.





Per definire la fattibilità di questo progetto ho studiato innanzi tutto l’architettura del software in uso con le annesse tecnologie principali: AngularJS e Spring.

Il passo successivo è stato quello di ricercare un metodo che permettesse di elaborare un testo preso in input e restituire in output una azione da compiere in base al contenuto dell’input. Ho deciso di utilizzare un particolare tipo di intelligenza artificiale chiamata ‘Natural Language Understanding’ (NLU), scegliendola per la sua capacità di comprensione della struttura e il significato da parte degli elaboratori del linguaggio umano (nel mio caso l’italiano) permettendo così agli utenti di interagire con i computer usando frasi di uso comune.

Per la creazione effettiva del prototipo avevo quindi bisogno di un software che eseguisse il lavoro di NLU e di una chat dove permettere uno scambio sia di tipo vocale che testuale con l’utente.

La prima versione dell’intelligenza artificiale è stata creata grazie allo strumento messo a disposizione da Google chiamato DialogFlow, il quale mi ha consentito di creare un primo modello di NLU. Purtroppo questo prototipo si è rivelato inutilizzabile per questioni di privacy in quanto i dati di natura strettamente personale e monetaria sarebbero stati processati da un server esterno all’azienda, 





<frase iniziale> 
<perchè ho scelto questa tesi>
<perchè edp ha scelto questo percorso>
<che progetto è?>

<spiegazione generica degli elementi fondamentali del progetto>
L'inizio del progetto \'e principalmente orientato allo studio delle componenti principali del software esistente e comprensione del pattern archietetturale Model-View-Controller presente in tutto il frontend.

Per iniziare a sviluppare un primo prototipo della chat la prima cosa da fare \'e stata studiare le tecnologie usate dalla piattaforma: JavaScript e AngularJs.  

Il progetto \'e iniziato con lo studio delle tecnologie: il patter architetturale Model-view-controller

Il sfrutta in maniera massiccia il pattern architetturale Model-view-controller

L'architettura di questo software si fonda sul concetto del Model-view-controller che \'e un pattern molto usato per sviluppare interfacce utente. Il MVC divide l'applicazione in 

Per poter cominciare a lavorare sul software \'e stato necessario uno studio approfondito di tutte le sue componenti: i linguaggi usati, i framework, i pattern principali e la suddivisione backend e frontend.

<definizione di un percorso iniziale>
<spiegazione dell'architettura del software di BPS(swisse)>
<studio della libreria principale usata dal frontend del software>
<studio degli elementi principali per la creazione del frontend ----> chat di testo>
<creazione della chat e test del prototipo usando sia testo che voce>
<studio di una possibile implementazione del backend di nlu>
<perchè è stato scelto di non usare dialogflow>
<passaggio a RASA>
<sviluppo del primo modello usato per il NLU>
<creazione del server rasa>
<refactor chat>
<perfezionamento del modello di NLU>
<integrazione>
<problemi affrontati durante l'integrazione>
<conclusione introduzione>
