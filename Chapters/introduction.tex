\chapter{Introduzione}

La tesi che presenter\'{o} tratter\`{a} del lavoro che ho svolto durante il tirocinio presso l'azienda fiorentina E.D.P. Service. 
L'obbiettivo del progetto \`{e} stato quello di creare un metodo per eseguire pagamenti vocali all'interno della piattaforma bancaria del cliente BPS(Suisse).
Per definire la fattibilit\'{a} di questo progetto ho studiato innanzi tutto l'architettura del software in uso con le annesse tecnologie principali: AngularJS e Spring.
Il passo successivo \'{e} stato quello di ricercare un metodo che permettesse di elaborare un testo preso in input e restituire in output una azione da compiere in base al contenuto dell'input. Le ricerche sono confluite in un particolare tipo di intelligenza artificiale chiamata Natural Language Understanding (che d'ora in poi indicher\'{o} con NLU), tradotto letteralmente comprensione del linguaggio naturale. 
Il NLU \`{e} definito come la comprensione della struttura e il significato da parte degli elaboratori del linguaggio umano (per esempio Inglese, Spagnolo, Giapponese) permettendo agli utenti di interagire con i computer usando frasi naturali.
Per la creazione effettiva del prototipo avevo quindi bisogno di un software che eseguisse il lavoro di NLU e una chat dove permettere uno scambio sia di tipo vocale che testuale con l'utente.
La prima versione dell'intelligenza artificiale \'e stata creata grazie allo strumento messo a disposizione da Google chiamato DialogFlow. Questo strumento mi ha consentito di creare un primo modello di NLU, comprendendo l'effettivo potenziale di un'intelligenza artificiale di questo tipo. 
Questo strumento per\'o non \'e stato scelto per lo sviluppo successivo per questioni di privacy, dato che i dati che sarebbero circolati nei server Google sarebbero stati dati di natura strettamente personale e monetaria.
Era quindi necessario un software da poter gestire e sviluppare internamente all'azienda, che fosse open source e in una versione stabile. Leggendo in vari forum e in vari articoli Rasa era quello che faceva al caso nostro.



Il mondo si sta evolvendo e la maggior parte degli utenti si aspetta che i software siano nuovi e all'avanguardia, questo \'e uno dei motivi che ha spinto l'azienda E.D.P. Service a proporre un progetto di tirocinio che riguardasse l'intelligenza artificiale. 

Negli ultimi anni l'interesse verso metodi vocali di interazione con ciò che ci circonda ha suscitato sempre più interesse

Negli ultimi anni l'interazione con le macchine ha visto un interesse sempre maggiore soprattutto 

Negli ultimi anni l'interazione con i dispositivi è cresciuta in maniera esponenziale, l'automazione dei processi quotidiani è da sempre 

L'interesse verso l'automazione dei processi quotidiani in questi ultimi anni sta avendo una crescita esponenziale, i servizi rilasciati da ogni azienda vede sempre più presente dei metodi automatici e veloci di 

L'interesse verso l'automazione dei processi quotidiani attivati dalla propria voce sta prendendo sempre più piede, 



Negli ultimi anni 


il mondo si sta evolvendo e la maggior parte degli utenti si aspetta che i software siano aggiornati e all’avanguardia, questo è stato il motivo per cui ho scelto di lavorare con software di intelligenza artificiale.

L’azienda E.D.P. Service, con cui ho collaborato per svolgere il tirocinio, mi ha proposto un progetto per lo sviluppo di un software che permettesse di eseguire pagamenti vocali all’interno della piattaforma bancaria della BPS(Suisse).

Per definire la fattibilità di questo progetto ho studiato innanzi tutto l’architettura del software in uso con le annesse tecnologie principali: AngularJS e Spring.

Il passo successivo è stato quello di ricercare un metodo che permettesse di elaborare un testo preso in input e restituire in output una azione da compiere in base al contenuto dell’input. Ho deciso di utilizzare un particolare tipo di intelligenza artificiale chiamata ‘Natural Language Understanding’ (NLU), scegliendola per la sua capacità di comprensione della struttura e il significato da parte degli elaboratori del linguaggio umano (nel mio caso l’italiano) permettendo così agli utenti di interagire con i computer usando frasi di uso comune.

Per la creazione effettiva del prototipo avevo quindi bisogno di un software che eseguisse il lavoro di NLU e di una chat dove permettere uno scambio sia di tipo vocale che testuale con l’utente.

La prima versione dell’intelligenza artificiale è stata creata grazie allo strumento messo a disposizione da Google chiamato DialogFlow, il quale mi ha consentito di creare un primo modello di NLU. Purtroppo questo prototipo si è rivelato inutilizzabile per questioni di privacy in quanto i dati di natura strettamente personale e monetaria sarebbero stati processati da un server esterno all’azienda, 





<frase iniziale> 
<perchè edp ha scelto questo percorso>
<perchè ho scelto questa tesi>
<che progetto è?>
<spiegazione generica degli elementi fondamentali del progetto>
<definizione di un percorso iniziale>
<spiegazione dell'architettura del software di BPS(swisse)>
<studio della libreria principale usata dal frontend del software>
<studio degli elementi principali per la creazione del frontend ----> chat di testo>
<creazione della chat e test del prototipo usando sia testo che voce>
<studio di una possibile implementazione del backend di nlu>
<perchè è stato scelto di non usare dialogflow>
<passaggio a RASA>
<sviluppo del primo modello usato per il NLU>
<creazione del server rasa>
<refactor chat>
<perfezionamento del modello di NLU>
<integrazione>
<problemi affrontati durante l'integrazione>
<conclusione introduzione>






