\chapter{Introduzione}

\section{Azienda e Software BPS}
Al giorno d'oggi l'utilizzo della voce per automatizzare i processi della vita quotidiana si sta sempre più espandendo, dall'assistente vocale nella propria casa a quello nel proprio cellulare. Sono sempre stato incuriosito da queste tecnologie e il tirocinio era il modo migliore per comprendere lo sviluppo di software di questo tipo. Il progetto mi è stato proposto dall'azienda fiorentina E.D.P. Service, società collegata al gruppo Best Vision Holding, il cui scopo era integrare una funzionalità nuova nel sito del cliente BPS(suisse) che permettesse di eseguire pagamenti o trasferimenti di denaro usando linguaggio naturale comunicando direttamente con l'applicazione.

\section{Analisi del problema}
Il software su cui avrei dovuto lavorare era una classica web application di monetica \footnote{Insieme delle tecniche di pagamento attraverso strumenti informatici e telematici}. Software di questo tipo si suddividono in 2 parti: una parte adibita all'interazione utente definita interfaccia grafica o frontend e la parte che gestisce le informazioni chiamata backend o lato server.
La parte del frontend gestisce principalmente le modalità di accesso dell'utente consentendogli di interagire con l'applicativo eseguendo le azioni messe a disposizione, come ad esempio: visualizzare il numero di transazioni eseguite, creare un pagamento, controllare l'andamento del mercato in base a una determinata valuta.
Per riuscire a sviluppare un'integrazione in questo sistema ho studiato i linguaggi usati dalla piattaforma della banca, principalmente formata da codice HTML, CSS, e AngularJS.
L'ultimo citato è un web-framework di casa Google scritto in JavaScript che permette di dinamicizzare una pagina web utilizzando dei comandi a livello di codice HTML, chiamati direttive, che vengono poi eseguiti da AngularJs una volta che la pagina web viene caricata dal browser.
Tutto ciò che riguarda il lato server: gestione, elaborazione e immagazzinamento dei dati forniti dal frontend, viene eseguita dal backend, parte fondamentale di un servizio web bancario che deve garantire sicurezza e reperibilità dei dati. Il backend sviluppato da Best Vision è interamente in linguaggio Java con framework Spring affiancato da un database per la persistenza dei dati.
I pagamenti compiuti tramite la piattaforma web utilizzano la classica procedura di compilazione manuale di alcuni campi di testo, come ad esempio la quantità di denaro da trasferire e l'iban del conto del destinatario.
Inserire una funzionalità che permettesse di eseguire queste azioni usando la sola voce avrebbe diminuito il numero di azioni che l'utente avrebbe dovuto fare per ottenere ciò che cercava. In poche parole quello che avrei dovuto creare era un software che imitasse il comportamento umano di un operatore di banca. Per fare questo è stato pensato di sviluppare un servizio di chatbot \footnote{Software che simula conversazioni naturali tra umani} che riuscisse a comprendere il linguaggio naturale umano e producesse una risposta in completa autonomia eseguendo le azioni richieste dall'utente.
Il pagamento tramite chat doveva quindi eseguire tutte le richieste al server nel giusto ordine prima di completare la transazione voluta: autorizzazione per accedere al conto dell'utente, richiesta di trasferimento di capitale verso terzi e delega della conferma della transazione.
L'architettura stilizzata del modulo che devo implementare è formata da 2 strutture: lato utente una chat dove è possibile scegliere se scrivere un testo con linguaggio naturale oppure utilizzare il servizio vocale, lato server un'intelligenza artificiale che sulla base dei messaggi dell'utente generi una risposta e agisca in base alle richieste.
Lo scopo di questa intelligenza artificiale è quello di riuscire a sostenere delle conversazioni con un utente reale nel modo più naturale possibile. Questo significa che quando un utente invia un messaggio la risposta generata deve essere il più possibile consona alla richiesta fatta, per ottenere un risultato di questo tipo deve esser fatta un'analisi del testo e conseguentemente formulato un messaggio di senso compiuto che riesca a rispondere in maniera corretta.
Solo una volta creato il prototipo standalone \footnote{Capacità di un componente di essere eseguito in maniera isolata e indipendente da agenti esterni} funzionante sarebbe stato possibile richiedere l'integrazione nel software di home banking aziendale.
