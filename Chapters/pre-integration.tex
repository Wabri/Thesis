\chapter{Pre-integrazione}

\section{Studio degli strumenti per la creazione di una chat stand alone}

\subsection{Linguaggi}

\subsection{Frameworks}

\subsection{Frontend della chat}

\section{Creazione dell'intelligenza artificiale}

\subsection{Google-Dialogflow}

\subsection{Rasa}

\subsection{Creazione data set per training}

\section{Aggiornamento Frontend per la comunicazione con Rasa}

\section{Sistema di pagamento vocale}

\section{Funzionamento e esecuzione}


<studio di javascript>
<studio di angularjs nel libro (vedi https://github.com/Wabri/UniversityInternship#day-01-020518--55-ore)>
<creazione di una chat testuale>
<integrazione di un sistema di speech recognition>
<inserimento della comunicazione tra la chat e dialogflow>
<vari test di funzionamento della comunicazione>
<passaggio da dialogflow a rasa>
<studio di rasa nlu>
<creazione di un primo modello di training con vari esempi esterni al progetto>
<processo di apprendimento del funzionamento effettivo di rasa lungo>
<tentativo di utilizzo del modello creato precedentemente, con dialogflow, per la creazione del modello di rasa nlu>
<passaggio di competenze(?) da dialogflow a rasa, eliminando la comunicazione effettiva con google>
<connessione con il server bps e rasa>
<per errore ho lasciato che la chat inviasse ancora i dati di test ai server di dialogflow = rasa>dialogflow>
<tutor universitario consiglia l'uso di typescript => studio di typescript => riscrittura della chat usando typescript>
<typescript migliore lettura del codice e possibilità di creazione di classi e oggetti>
<estrapolazione dei dati utente dal server della banca usando chiamate rest>
<generazione data set dell'intelligenza artificiale>
<necessario un server che eseguisse azioni richieste>
<studiare python per poter creare esempi di rasa core>
<studio di rasa core - server>
<utilizzo di chiamate rest all'interno del chatbot verso rasa core>
<problemi con la messa in opera del server su macchina windows>
<utilizzo del server su rete locale per i test>
<refactor del codice chatbot usando superagent (vedi https://github.com/Wabri/UniversityInternship#day-26-110718--55-ore)>
<iniziato a creare le story per generare un modello efficace di intelligenza>
<Cominciato a generare le prime azioni di risposta in base agli intenti e al contenuto del testo fornito in input>
<Creato l'effettivo bot scritto in python, dove risiedono tutte le azioni che può fare il bot>
<Migliorato il modello di training aggiungendo nuovi esempi di intents e stories>
<problemi con l'autenticazione a 2 fattori per il prototipo, non riesce a estrarre direttamente user e password e devo inserirle manualmente con chiamate post verso il server rasa>
<completamento comunicazione user-rasafrontend-rasabackend-backendspringbanca - vedi https://github.com/Wabri/UniversityInternship/blob/master/README.md#day-36-020818--65-ore>
<creazione delle azioni effettive che deve fare il bot rasa backend>
<il problema del xcsrf token e jsession>
<vedi https://github.com/Wabri/UniversityInternship/blob/master/README.md#day-38-210818--75-ore per maggiori informazioni>
<documentazione prototipo>















