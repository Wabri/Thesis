\chapter{Pre-integrazione}

\section{Studio degli strumenti per la creazione di una chat stand alone}

\subsection{Linguaggi}

\subsection{Frameworks}

\subsection{Frontend della chat}

\section{Creazione dell'intelligenza artificiale}

\subsection{Google-Dialogflow}

\subsection{Rasa}

\subsection{Creazione data set per training}

\section{Aggiornamento Frontend per la comunicazione con Rasa}

\section{Sistema di pagamento vocale}

\section{Funzionamento e esecuzione}

\iffalse
<studio di javascript>
\fi
Per creare una chat nei moderni siti web dinamici \'e stato necessario lo studio di un linguaggio che sicuramente ha modernizzato lo sviluppo web cio\'e JavaScript soprattutto per la quantit\'a impressionante di framework sviluppati con questa tecnologia.
\'E un linguaggio interpretato, che ha molto poco a che fare con il quasi omonimo Java, la prima versione venne rilasciata nel 1995 con l'obbiettivo di arricchire le pagine web aggiungendo alcune forme di dinamicit\'a. L'utilizzo in una pagina \'e possibile solo se il browser del client possiede un supporto a interpretare il linguaggio, questa scelta di mantenere la responsabilit\'a a livello client \'e stata fatta per non sovracaricare troppo il server che riliasciava il servizio web relativo.
 
\iffalse 
<studio di angularjs nel libro (vedi https://github.com/Wabri/UniversityInternship#day-01-020518--55-ore)>
\fi
Le prime fasi di tirocinio le ho concentrate sull'apprendimento dei vari strumenti necessari per comprendere l'architettura della web application formata principalmente da codice HTML con direttive AngularJS e Javascript, tenuti insieme dal MVC (Model View Controller).
Il MVC \'e un pattern architetturale basato su 3 componenti che hanno lo scopo di usare e gestire i dati con scopi differenti: il model deve gestire i dati e fornire metodi per usarli nonch\'e la logica e le regole dell'applicazione, il view ha lo scopo di rappresentare i dati sotto forma di informazione in modo tale da interagire con l'utente che li visualizzer\'a, e infine il controller che attraverso istruzioni fornite dall'utente modifica lo stato degli altri 2 componenti modificando i dati gestiti dal controller o indicando una modalit\'a differente di visualizzazione dei dati al view.
\iffalse
metti st'immagine o una simile
https://it.wikipedia.org/wiki/File:MVC-Process.png
\fi
Questa struttura viene usata principalmente per dividere la logica di business con l'interfaccia utente, migliorando quindi l'organizzazione dei processi interni di una application. 
L'utilizzo di questo pattern direttamente sul client ha permesso un cambiamento importante per quanto riguarda le dynamic pages, cio\'e quello di non eseguire un redirect per modificare la pagina ma eseguendo chiamate asincrone al server. 
Con il tempo \'e stato necessario fornire un framework che implementasse queste logiche riducendo i tempi di sviluppo andando a generalizzare tutti quei comportamenti classici del pattern MVC, se si parla di JavaScript senza alcun dubbio il framework pi\'u usato \'e AngularJs.
Questa libreria funziona per mezzo di comandi definiti direttive che vengono inserite direttamente nel codice HTML che permettono di eseguire una sorta di comunicazione tra le varie parti del client. Il framework viene caricato all'interno della pagina attraverso l'import definito dal tag <script>:
\lstset{frame=tb,
  language=HTML,
  aboveskip=3mm,
  belowskip=3mm,
  showstringspaces=false,
  columns=flexible,
  basicstyle={\small\ttfamily},
  numbers=none,
  numberstyle=\tiny\color{gray},
  keywordstyle=\color{blue},
  commentstyle=\color{dkgreen},
  stringstyle=\color{mauve},
  breaklines=true,
  breakatwhitespace=true,
  tabsize=3
}
\begin{lstlisting}
<script 
 src="https://ajax.googleapis.com/ajax/libs/angularjs/1.2.19/angular.js">
</script>
\end{lstlisting}
Questo 
\iffalse
<creazione di una chat testuale>
<integrazione di un sistema di speech recognition>
<inserimento della comunicazione tra la chat e dialogflow>
<vari test di funzionamento della comunicazione>
<passaggio da dialogflow a rasa>
<studio di rasa nlu>
<creazione di un primo modello di training con vari esempi esterni al progetto>
<processo di apprendimento del funzionamento effettivo di rasa lungo>
<tentativo di utilizzo del modello creato precedentemente, con dialogflow, per la creazione del modello di rasa nlu>
<passaggio di competenze(?) da dialogflow a rasa, eliminando la comunicazione effettiva con google>
<connessione con il server bps e rasa>
<per errore ho lasciato che la chat inviasse ancora i dati di test ai server di dialogflow = rasa>dialogflow>
<tutor universitario consiglia l'uso di typescript => studio di typescript => riscrittura della chat usando typescript>
<typescript migliore lettura del codice e possibilità di creazione di classi e oggetti>
<estrapolazione dei dati utente dal server della banca usando chiamate rest>
<generazione data set dell'intelligenza artificiale>
<necessario un server che eseguisse azioni richieste>
<studiare python per poter creare esempi di rasa core>
<studio di rasa core - server>
<utilizzo di chiamate rest all'interno del chatbot verso rasa core>
<problemi con la messa in opera del server su macchina windows>
<utilizzo del server su rete locale per i test>
<refactor del codice chatbot usando superagent (vedi https://github.com/Wabri/UniversityInternship#day-26-110718--55-ore)>
<iniziato a creare le story per generare un modello efficace di intelligenza>
<Cominciato a generare le prime azioni di risposta in base agli intenti e al contenuto del testo fornito in input>
<Creato l'effettivo bot scritto in python, dove risiedono tutte le azioni che può fare il bot>
<Migliorato il modello di training aggiungendo nuovi esempi di intents e stories>
<problemi con l'autenticazione a 2 fattori per il prototipo, non riesce a estrarre direttamente user e password e devo inserirle manualmente con chiamate post verso il server rasa>
<completamento comunicazione user-rasafrontend-rasabackend-backendspringbanca - vedi https://github.com/Wabri/UniversityInternship/blob/master/README.md#day-36-020818--65-ore>
<creazione delle azioni effettive che deve fare il bot rasa backend>
<il problema del xcsrf token e jsession>
<vedi https://github.com/Wabri/UniversityInternship/blob/master/README.md#day-38-210818--75-ore per maggiori informazioni>
<documentazione prototipo>
\fi