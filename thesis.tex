%--------------------------------------------------------------
% thesis.tex
%--------------------------------------------------------------
% Corso di Laurea in Informatica
% http://if.dsi.unifi.it/
% @Facolt\`a di Scienze Matematiche, Fisiche e Naturali
% @Universit\`a degli Studi di Firenze
%--------------------------------------------------------------
% - template for the main file of Informatica@Unifi Thesis
% - based on Classic Thesis Style Copyright (C) 2008
%   Andrè Miede http://www.miede.de
%--------------------------------------------------------------
\documentclass[twoside,openright,titlepage,fleqn,
	headinclude,12pt,a4paper,BCOR5mm,footinclude]{scrbook}
%--------------------------------------------------------------
\newcommand{\myItalianTitle}{Integrazione di un prototipo chatbot vocale per pagamenti bancari\xspace}
% use the right myDegree option
\newcommand{\myDegree}{Corso di Laurea Triennale in Informatica\xspace}
\newcommand{\myName}{Gabriele Puliti\xspace}
\newcommand{\myProf}{Lorenzo Bettini\xspace}
\newcommand{\myOtherProf}{Correlatore\xspace}
\newcommand{\mySupervisor}{Tommaso Tamantini\xspace}
\newcommand{\myFaculty}{
	Scuola di Scienze Matematiche, Fisiche e Naturali\xspace}
\newcommand{\myUni}{\protect{
	Universit\`a degli Studi di Firenze}\xspace}
\newcommand{\myLocation}{Firenze\xspace}
\newcommand{\myTime}{Anno Accademico 2018-2019\xspace}
\newcommand{\myVersion}{Version 0.1\xspace}
%--------------------------------------------------------------
\usepackage[italian]{babel}
\usepackage[utf8x]{inputenc}
\usepackage[T1]{fontenc}
\usepackage[square,numbers]{natbib}
\usepackage[fleqn]{amsmath}
\usepackage{ellipsis}
\usepackage{listings}
\usepackage{subfig}
\usepackage{caption}
\usepackage{appendix}
\usepackage{siunitx}
\usepackage{natbib}
\usepackage{listings} %Per inserire codice\usepackage{listings}
\usepackage{color}
\usepackage{graphicx}
\usepackage{wrapfig}
\usepackage{float}
\definecolor{dkgreen}{rgb}{0,0.6,0}
\definecolor{gray}{rgb}{0.5,0.5,0.5}
\definecolor{mauve}{rgb}{0.58,0,0.82}


%--------------------------------------------------------------
\usepackage{dia-classicthesis-ldpkg}
%--------------------------------------------------------------
% Options for classicthesis.sty:
% tocaligned eulerchapternumbers drafting linedheaders
% listsseparated subfig nochapters beramono eulermath parts
% minionpro pdfspacing
\usepackage[eulerchapternumbers,linedheaders,subfig,beramono,eulermath,
parts]{classicthesis}
%--------------------------------------------------------------
\newlength{\abcd} % for ab..z string length calculation
% how all the floats will be aligned
\newcommand{\myfloatalign}{\centering}
\setlength{\extrarowheight}{3pt} % increase table row height
\captionsetup{format=hang,font=small}
%--------------------------------------------------------------
% Layout setting
%--------------------------------------------------------------
\usepackage{geometry}
\geometry{
	a4paper,
	ignoremp,
	bindingoffset = 1cm,
	textwidth     = 13.5cm,
	textheight    = 21.5cm,
	lmargin       = 3.5cm, % left margin
	tmargin       = 4cm    % top margin
}

\colorlet{punct}{red!60!black}
\definecolor{delim}{RGB}{20,105,176}
\colorlet{numb}{magenta!60!black}
\lstdefinelanguage{json}{
    numbers=left,
    numberstyle=\scriptsize,
    stepnumber=1,
    numbersep=8pt,
    showstringspaces=false,
    breaklines=true,
    frame=lines,
    literate=
     *{0}{{{\color{numb}0}}}{1}
      {1}{{{\color{numb}1}}}{1}
      {2}{{{\color{numb}2}}}{1}
      {3}{{{\color{numb}3}}}{1}
      {4}{{{\color{numb}4}}}{1}
      {5}{{{\color{numb}5}}}{1}
      {6}{{{\color{numb}6}}}{1}
      {7}{{{\color{numb}7}}}{1}
      {8}{{{\color{numb}8}}}{1}
      {9}{{{\color{numb}9}}}{1}
      {:}{{{\color{punct}{:}}}}{1}
      {,}{{{\color{punct}{,}}}}{1}
      {\{}{{{\color{delim}{\{}}}}{1}
      {\}}{{{\color{delim}{\}}}}}{1}
      {[}{{{\color{delim}{[}}}}{1}
      {]}{{{\color{delim}{]}}}}{1},
}

\lstdefinelanguage{JavaScript}{
    keywords={typeof, new, true, false, catch, function, return, null, catch, const, switch, var, if, in, while, do, else, case, break},
    keywordstyle=\color{blue}\bfseries,
    ndkeywords={class, export, boolean, throw, implements, import, require, this},
    numbers=left,
    numberstyle=\scriptsize,
    stepnumber=1,
    numbersep=8pt,
    showstringspaces=false,
    breaklines=true,
    frame=lines,
    ndkeywordstyle=\color{darkgray}\bfseries,
    identifierstyle=\color{black},
    sensitive=false,
    comment=[l]{//},
    morecomment=[s]{/*}{*/},
    commentstyle=\color{purple}\ttfamily,
    stringstyle=\color{red}\ttfamily,
    morestring=[b]',
    morestring=[b]"
}

\lstdefinelanguage{betterHtml}{
    language=html,
    sensitive=true, 
    alsoletter={<>=-},
    otherkeywords={
    % HTML tags
    <html>, <head>, <title>, </title>, <meta, />, </head>, <body>,
    <canvas, \/canvas>, <script>, <script, </script>, </body>, </html>, <!, html>, <style>, </style>, ><
    },  
    ndkeywords={
    % General
    =,
    % HTML attributes
    charset=, id=, width=, height=,
    % CSS properties
    border:, transform:, -moz-transform:, transition-duration:, transition-property:, transition-timing-function:
    },  
    morecomment=[s]{<!--}{-->},
    tag=[s]
    numbers=left,
    stepnumber=1,
    showstringspaces=false,
    breaklines=true,
    frame=lines,
    identifierstyle=\color{black},
    sensitive=false,
    commentstyle=\color{purple}\ttfamily,
    stringstyle=\color{red}\ttfamily,
}

\newcommand\YAMLcolonstyle{\color{red}}
\newcommand\YAMLkeystyle{\color{black}}
\newcommand\YAMLvaluestyle{\color{blue}}
\makeatletter
% here is a macro expanding to the name of the language
% (handy if you decide to change it further down the road)
\newcommand\language@yaml{yaml}
\expandafter\expandafter\expandafter\lstdefinelanguage
\expandafter{\language@yaml}
{
  numbers=left,
  numberstyle=\scriptsize,
  stepnumber=1,
  numbersep=8pt,
  breaklines=true,
  frame=lines,
  keywords={true,false,null,y,n},
  keywordstyle=\color{darkgray}\bfseries,
  basicstyle=\YAMLkeystyle,                                 % assuming a key comes first
  sensitive=false,
  comment=[l]{\#},
  morecomment=[s]{/*}{*/},
  commentstyle=\color{purple}\ttfamily,
  stringstyle=\YAMLvaluestyle\ttfamily,
  moredelim=[l][\color{orange}]{\&},
  moredelim=[l][\color{magenta}]{*},
  moredelim=**[il][\YAMLcolonstyle{:}\YAMLvaluestyle]{:},   % switch to value style at :
  morestring=[b]',
  morestring=[b]",
  literate =    {---}{{\ProcessThreeDashes}}3
                {>}{{\textcolor{red}\textgreater}}1     
                {|}{{\textcolor{red}\textbar}}1 
                {\ -\ }{{\mdseries\ -\ }}3,
}
% switch to key style at EOL
\lst@AddToHook{EveryLine}{\ifx\lst@language\language@yaml\YAMLkeystyle\fi}
\makeatother
\newcommand\ProcessThreeDashes{\llap{\color{cyan}\mdseries-{-}-}}

%--------------------------------------------------------------
\begin{document}
\frenchspacing
\raggedbottom
\pagenumbering{roman}
\pagestyle{plain}
%--------------------------------------------------------------
% Frontmatter
%--------------------------------------------------------------
\include{titlePage}
\pagestyle{scrheadings}
%--------------------------------------------------------------
% Mainmatter
%--------------------------------------------------------------
\pagenumbering{arabic}
% use \cleardoublepage here to avoid problems with pdfbookmark
%\include{intro} % use \myChapter command instead of \chapter
\tableofcontents
%\listoffigures
\cleardoublepage
\thispagestyle{empty}
\begin{flushright}
\null\vspace{\stretch {1}}
\emph{"Dedicata a mike" \break --- Bindi} \vspace{\stretch{2}}\null
\end{flushright}
\chapter{Introduzione}
Al giorno d'oggi l'utilizzo della voce per automatizzare i processi della vita quotidiana si sta sempre pi\'u espandendo, dall'assistente vocale nella propria casa a quello nel proprio cellulare. Sono sempre stato incuriosito da questo tipo di tecnologie e il tirocinio era il modo migliore per comprendere lo sviluppo di software di questo tipo. Il progetto mi \'e stato proposto dall'azienda fiorentina E.D.P. Service, societ\'a collegata al gruppo Best Vision, il cui scopo era creare un prototipo di chat vocale per pagamenti bancari da integrare poi nel sito del cliente BPS(suisse).
L'azienda mi ha proposto un progetto per lo sviluppo di un software che permettesse di eseguire pagamenti vocali all'interno della piattaforma web della BPS(Suisse).
Software di questo tipo si suddividono in 2 parti: una parte visibile all'utente con cui pu\'o interagire rappresentata dal frontend (chiamato anche interfaccia grafica) e la parte che gestisce le informazioni rappresentata dal backend (chiamata anche lato server).
La parte del frontend gestisce principalmente le modalit\'a di accesso dell'utente consentendogli di interagire con l'applicativo per eseguire la azioni messe a disposizione, come ad esempio: visualizzare il numero di transazioni eseguite, creare un pagamento, controllare l'andamento del mercato in base a una determinata valuta. Inserire una chat che permettesse di eseguire queste azioni usando la sola voce avrebbe diminuito il numero di azioni che l'utente avrebbe dovuto eseguire per ottenere ci\'o che cercava. 
Per poterlo fare ho studiato i linguaggi usati dalla piattaforma che principalmente era formata da codice HTML, CSS e AngularJS, web-framework di casa Google scritto in JavaScript la cui prima versione fu rilasciata nel 2010. Questa libreria permette di dinamicizzare una pagina web utilizzando dei comandi, chiamati direttive, a livello di codice HTML che vengono poi eseguiti da AngularJS una volta che la pagina viene caricata.
Tutto ci\'o che riguarda il lato server: gestione, elaborazione e immagazzinamento dei dati forniti dal frontend, viene eseguita dal backend parte fondamentale di un servizio web bancario che deve garantire sicurezza e reperibilità dei dati. Il backend sviluppato da Best Vision \'e diviso in un servizio di database e uno di gestione del database scritto per intero in linguaggio Java usando il framework Spring. Il pagamento tramite chat doveva quindi eseguire tutte le richieste al backend nel giusto ordine prima di completare la transazione voluta: autorizzazione dell'utente alla gestione del conto, richiesta di trasferimento di capitale verso un altro utente e delega della conferma del pagamento.
Quando un utente invia un messaggio si aspetta che venga generata una risposta consona, cioè venga eseguita una analisi del testo e formulata una frase di senso compiuto che riesca a rispondere in maniera corretta al messaggio dell'utente. La comprensione del testo viene definita comprensione del linguaggio naturale, in inglese Natural Language Understanding (d'ora in poi lo indicherò con NLU), cio\'e attribuire alla frase analizzata un intento e estrarre da essa oggetti che possono essere utili per la generazione di una risposta.

Se il lavoro di comprensione e generazione della frase di risposta deve essere fatto da un computer questo non \'e semplice
Il problema principale della creazione di un bot \'e la creazione di un metodo di comprensione del testo che l'utente ha inviato



Per definire la fattibilit\'{a} di questo progetto ho studiato innanzi tutto l'architettura del software in uso con le annesse tecnologie principali: AngularJS e Spring.
Il passo successivo \'{e} stato quello di ricercare un metodo che permettesse di elaborare un testo preso in input e restituire in output una azione da compiere in base al contenuto dell'input. Le ricerche sono confluite in un particolare tipo di intelligenza artificiale chiamata Natural Language Understanding (che d'ora in poi indicher\'{o} con NLU), tradotto letteralmente comprensione del linguaggio naturale. 
Il NLU \`{e} definito come la comprensione della struttura e il significato da parte degli elaboratori del linguaggio umano (per esempio Inglese, Spagnolo, Giapponese) permettendo agli utenti di interagire con i computer usando frasi naturali.
Per la creazione effettiva del prototipo avevo quindi bisogno di un software che eseguisse il lavoro di NLU e una chat dove permettere uno scambio sia di tipo vocale che testuale con l'utente.
La prima versione dell'intelligenza artificiale \'e stata creata grazie allo strumento messo a disposizione da Google chiamato DialogFlow. Questo strumento mi ha consentito di creare un primo modello di NLU, comprendendo l'effettivo potenziale di un'intelligenza artificiale di questo tipo. 
Questo strumento per\'o non \'e stato scelto per lo sviluppo successivo per questioni di privacy, dato che i dati che sarebbero circolati nei server Google sarebbero stati dati di natura strettamente personale e monetaria.
Era quindi necessario un software da poter gestire e sviluppare internamente all'azienda, che fosse open source e in una versione stabile. Leggendo in vari forum e in vari articoli Rasa era quello che faceva al caso nostro.





Per definire la fattibilità di questo progetto ho studiato innanzi tutto l’architettura del software in uso con le annesse tecnologie principali: AngularJS e Spring.

Il passo successivo è stato quello di ricercare un metodo che permettesse di elaborare un testo preso in input e restituire in output una azione da compiere in base al contenuto dell’input. Ho deciso di utilizzare un particolare tipo di intelligenza artificiale chiamata ‘Natural Language Understanding’ (NLU), scegliendola per la sua capacità di comprensione della struttura e il significato da parte degli elaboratori del linguaggio umano (nel mio caso l’italiano) permettendo così agli utenti di interagire con i computer usando frasi di uso comune.

Per la creazione effettiva del prototipo avevo quindi bisogno di un software che eseguisse il lavoro di NLU e di una chat dove permettere uno scambio sia di tipo vocale che testuale con l’utente.

La prima versione dell’intelligenza artificiale è stata creata grazie allo strumento messo a disposizione da Google chiamato DialogFlow, il quale mi ha consentito di creare un primo modello di NLU. Purtroppo questo prototipo si è rivelato inutilizzabile per questioni di privacy in quanto i dati di natura strettamente personale e monetaria sarebbero stati processati da un server esterno all’azienda, 





<frase iniziale> 
<perchè ho scelto questa tesi>
<perchè edp ha scelto questo percorso>
<che progetto è?>

<spiegazione generica degli elementi fondamentali del progetto>
L'inizio del progetto \'e principalmente orientato allo studio delle componenti principali del software esistente e comprensione del pattern archietetturale Model-View-Controller presente in tutto il frontend.

Per iniziare a sviluppare un primo prototipo della chat la prima cosa da fare \'e stata studiare le tecnologie usate dalla piattaforma: JavaScript e AngularJs.  

Il progetto \'e iniziato con lo studio delle tecnologie: il patter architetturale Model-view-controller

Il sfrutta in maniera massiccia il pattern architetturale Model-view-controller

L'architettura di questo software si fonda sul concetto del Model-view-controller che \'e un pattern molto usato per sviluppare interfacce utente. Il MVC divide l'applicazione in 

Per poter cominciare a lavorare sul software \'e stato necessario uno studio approfondito di tutte le sue componenti: i linguaggi usati, i framework, i pattern principali e la suddivisione backend e frontend.

<definizione di un percorso iniziale>
<spiegazione dell'architettura del software di BPS(swisse)>
<studio della libreria principale usata dal frontend del software>
<studio degli elementi principali per la creazione del frontend ----> chat di testo>
<creazione della chat e test del prototipo usando sia testo che voce>
<studio di una possibile implementazione del backend di nlu>
<perchè è stato scelto di non usare dialogflow>
<passaggio a RASA>
<sviluppo del primo modello usato per il NLU>
<creazione del server rasa>
<refactor chat>
<perfezionamento del modello di NLU>
<integrazione>
<problemi affrontati durante l'integrazione>
<conclusione introduzione>

\chapter{Pre-integrazione}

\section{Studio degli strumenti per la creazione di una chat stand alone}

\subsection{Linguaggi}

\subsection{Frameworks}

\subsection{Frontend della chat}

\section{Creazione dell'intelligenza artificiale}

\subsection{Google-Dialogflow}

\subsection{Rasa}

\subsection{Creazione data set per training}

\section{Aggiornamento Frontend per la comunicazione con Rasa}

\section{Sistema di pagamento vocale}

\section{Funzionamento e esecuzione}


<studio di javascript>
<studio di angularjs nel libro (vedi https://github.com/Wabri/UniversityInternship#day-01-020518--55-ore)>
<creazione di una chat testuale>
<integrazione di un sistema di speech recognition>
<inserimento della comunicazione tra la chat e dialogflow>
<vari test di funzionamento della comunicazione>
<passaggio da dialogflow a rasa>
<studio di rasa nlu>
<creazione di un primo modello di training con vari esempi esterni al progetto>
<processo di apprendimento del funzionamento effettivo di rasa lungo>
<tentativo di utilizzo del modello creato precedentemente, con dialogflow, per la creazione del modello di rasa nlu>
<passaggio di competenze(?) da dialogflow a rasa, eliminando la comunicazione effettiva con google>
<connessione con il server bps e rasa>
<per errore ho lasciato che la chat inviasse ancora i dati di test ai server di dialogflow = rasa>dialogflow>
<tutor universitario consiglia l'uso di typescript => studio di typescript => riscrittura della chat usando typescript>
<typescript migliore lettura del codice e possibilità di creazione di classi e oggetti>
<estrapolazione dei dati utente dal server della banca usando chiamate rest>
<generazione data set dell'intelligenza artificiale>
<necessario un server che eseguisse azioni richieste>
<studiare python per poter creare esempi di rasa core>
<studio di rasa core - server>
















\chapter{Integrazione}

L'architettura dove avrei dovuto lavorare è formata da uno stack classico con spring che si occupa del lato backend e angularjs che invece gestisce il livello frontend, tutta la codebase viene versionata usando Subversion. Subversion è un version control system di tipo centralizzato, questo significa che si ha un server centrale che contiene tutto il codice. Il problema di usare un versionamento di questo tipo è che per poter sviluppare codice si deve essere collegati al server remoto altrimenti non è possibile apportare modifiche.
\begin{figure}[H]
    \centering
    \includegraphics[width=0.6\textwidth]{img/centralized.png}
    \caption{Centralized Version Control System}
\end{figure}
Alcune dipendenze lato frontend non possono essere usate, come socket.io, oppure non servono più, come express perchè il sistema in cui andiamo ad aggiungere questo modulo già prevede la creazione di uno web server. Per quanto riguarda socket.io ho dovuto riscrivere il codice andando a sostituire la libreria con \textbf{angular-socket-io}, un porting sotto forma di componente angularjs in modo da poter ricreare lo scambio di messaggi instaurato nel prototipo.
Il modulo è risultato essere composto da 3 elementi: 
\begin{itemize}
  \item \textbf{chatbot.html} la versione minimale dell'interfaccia web del prototipo,
  \item \textbf{chatbotUtilityService.js} script di utility in cui vengono definite le funzioni per comunicare con il server rasa,
  \item \textbf{chatbotController.js} la cui responsabilità è quella di gestire la comunicazione con il server rasa e modificare l'interfaccia grafica con la risposta
\end{itemize}
Il motivo per cui l'interfaccia è stata creata minimale è che per poter applicare una modifica grafica sono necessari degli studi di user experience per capire come renderla il più facile possibile da usare per l'utente. Considerando che la web app può essere visualizzata da tantissimi dispositivi diversi come cellulari di diversa forma o tablet e anche computer, il lavoro è stato demandato al reparto specializzato che avrebbe poi stabilito come questa sarebbe dovuta essere fatta.
Lo script di utility è stato creato per estrarre dal controller molte delle responsabilità che aveva, in modo tale da rendere più snella la leggibilità dei componenti e più facile la manutenzione. In questo componente si possono trovare tutte le funzioni che sono necessarie per la comunicazione in entrata e uscita verso il server dove si trova l'intelligenza artificiale.
Per poter utilizzare in produzione l'intelligenza artificiale è stato creato un server linux dedicato con tutte le dipendenze necessarie ad eseguire le fasi di preparazione e running di Rasa. Questo permette di rimuovere la complessità di creazione di nuovi modelli nella macchina locale di sviluppo rendendo più agile il lavoro di manutenzione e aggiornamento delle nuove versioni.
\begin{figure}[H]
 \centering
  \includegraphics[width=0.6\textwidth]{img/old-new.png}
 \caption{Vecchia e nuova architettura}
\end{figure}
Rimane comunque il problema del dataset con cui viene generato il modello dell'intelligenza artificiale, come è stato già detto, più stories abbiamo e più l'intelligenza è migliore. Ci sono due modi per creare nuove conversazioni: rilasciare una versione beta della chat in modo da ottenere molte conversazioni reali, oppure creare un team aziendale che rimpiazza il cliente inserendo quante più conversazioni differenti possibili.
Questa prima versione del chatbot ha la capacità esclusiva di listare i conti correnti di un cliente ed eseguire dei pagamenti preimpostati, la scelta di limitare le funzionalità è dovuta ad un fatto di sicurezza perchè ancora i modelli generati non erano abbastanza sicuri da poter essere usati in ambiente di produzione.

\input{Chapters/conclusion.tex}
\chapter{Conclusioni e sviluppi futuri}

Questo progetto ha dimostrato come integrare un chatbot all'interno di una web application già strutturata sia possibile anche se sono necessarie delle introduzioni di altre tecnologie non preesistenti. Nonostante le tecnologie usate non siano le più aggiornate, come ad esempio AngularJs \footnote{Che nel dicembre 2021 non verrà più mantenuto e ad oggi sono state create 11 versioni stabili dopo questa}, o le più facili da usare come il sistema di versionamento Subversion \footnote{Version Control system di tipo centralizzato, che ad oggi è stato sorpassato di gran lunga da Git che invece è di tipo distribuito}, è comunque possibile trovare degli workaround per poter introdurre nuove tecnologie all'interno di un software complesso come quello su cui ho dovuto lavorare.
L'utilizzo nella prima parte del progetto di Dialogflow è stato fondamentale per comprendere il tipo di software che era necessario creare, nonostante sia stato rimpiazzato poi da una tecnologia completamente differente e se vogliamo più complessa in termini di implementazione.
Il problema più grosso che ho riscontrato durante la creazione del prototipo è stato quello di non riuscire ad automatizzare il sistema di acquisizione dei token del cliente che però non sussiste nel caso in cui la chat sia già integrata, infatti in questo caso i due token possono essere inviati dal frontend non appena la chat viene aperta o non appena viene inviato il primo messaggio dal cliente.
Lo scopo finale del progetto è stato raggiunto portando una prima versione funzionante del chatbot richiesto. Gli sviluppi futuri di questa prima versione sono numerosi. Prima fra tutti è la necessità di migliorare la qualità delle stories che si trovano nel dataset di training dell'intelligenza in modo tale da rendere il modello più accurato e preciso. L'interfaccia della chat integrata al software è da rivedere aggiungendo migliorie grafiche studiate per rendere l'esperienza utente il più accogliente possibile. Infine il bot al momento ha solo le funzionalità base, ma è possibile implementare nuove funzionalità aggiornando il backend Rasa con tutte le operazioni che potrebbe fare un cliente della banca.
Il prototipo che ho creato è standalone e open source, l'intero codice infatti può essere visionato su Github \footnote{Servizio di hosting di progetti basato sulla tecnologia Git}, per questioni ovvie non si trova nessun riferimento diretto a codice e indirizzi del software proprietario aziendale.

\begin{thebibliography}{99}

\bibitem{Differenza tra frontend e backend}{Thapliyal, Vimal - \emph{Difference Between Frontend and Backend MVC – Joomlatuts}}

\bibitem{etichetta2}{Autore - \emph{Titolo} - altre informazioni}

\bibitem{etichetta2}{Autore - \emph{Titolo} - altre informazioni}

\end{thebibliography}


%--------------------------------------------------------------
\end{document}
%--------------------------------------------------------------
